% David Cain
% 2012-04-23
% Based off a template by Ted Pavlic from http://www.vel.co.nz

\documentclass[english]{report}
\usepackage{Tabbing}
\usepackage{amsmath,amsfonts,amsthm,amssymb}
\usepackage{chngpage}
\usepackage{extramarks}
\usepackage{fancyhdr}
\usepackage{graphicx,float,wrapfig}
\usepackage{isodate}
\usepackage{lastpage}
\usepackage{listings}
\usepackage{setspace}
\usepackage{soul,color}

% Use ISO-8601 for today's date
\isodate

% Configure listings for Python
\lstset{
  language=Python,
  showstringspaces=false,
  formfeed=\newpage,
  tabsize=4,
  commentstyle=\itshape
}

\newcommand{\code}[1]{
  %\hrulefill
  \lstinputlisting{#1}
  \vspace{2em}
}


% In case you need to adjust margins:
\topmargin=-0.45in
\evensidemargin=0in
\oddsidemargin=0in
\textwidth=6.5in
\textheight=9.0in
\headsep=0.25in

% Homework-specific information
\newcommand{\thedate}{2012-04-23}
\newcommand{\hmwkTitle}{Final Project Proposal}
\newcommand{\courseName}{CS\ 365}
\newcommand{\courseFullName}{Computer\ Vision}
\newcommand{\classTime}{11:00}
\newcommand{\courseInstructor}{Brian\ Eastwood}
\newcommand{\authorName}{David\ Cain,\ Justin\ Sperry}

% Setup the header and footer
\pagestyle{fancy}
\lhead{\authorName}
\chead{\courseName\ \hmwkTitle}
\rhead{\thedate}
\lfoot{\lastxmark}
\cfoot{}
\rfoot{Page\ \thepage\ of\ \pageref{LastPage}}
\renewcommand\headrulewidth{0.4pt}
\renewcommand\footrulewidth{0.4pt}

% This is used to trace down (pin point) problems
% in latexing a document:
%\tracingall

%%%%%%%%%%%%%%%%%%%%%%%%%%%%%%%%%%%%%%%%%%%%%%%%%%%%%%%%%%%%%
% Formatting
\newcommand{\enterProblemHeader}[1]{\nobreak\extramarks{#1}{#1 continued on next page\ldots}\nobreak%
                                    \nobreak\extramarks{#1 (continued)}{#1 continued on next page\ldots}\nobreak}%
\newcommand{\exitProblemHeader}[1]{\nobreak\extramarks{#1 (continued)}{#1 continued on next page\ldots}\nobreak%
                                   \nobreak\extramarks{#1}{}\nobreak}%
\setcounter{secnumdepth}{0}
\newcommand{\problemName}{}%
\newcounter{problemCounter}%
\newenvironment{problem}[1][\arabic{problemCounter}.)]%
  {\stepcounter{problemCounter}%
   \renewcommand{\problemName}{#1}%
   \section{\problemName}%
   \enterProblemHeader{\problemName}}%
  {\exitProblemHeader{\problemName}}%

\newcommand{\problemQuestion}[1]
  {\noindent\fbox{\begin{minipage}[c]{\columnwidth}#1\end{minipage}}}%
%%%%%%%%%%%%%%%%%%%%%%%%%%%%%%%%%%%%%%%%%%%%%%%%%%%%%%%%%%%%%

%%%%%%%%%%%%%%%%%%%%%%%%%%%%%%%%%%%%%%%%%%%%%%%%%%%%%%%%%%%%%
% Make title
\title{\vspace{2in}\textmd{\textbf{\hmwkClass:\
\hmwkTitle}}\\\normalsize\vspace{0.1in}\small{Due\ on\
\today}\\\vspace{0.1in}\large{\textit{\classInstructor\ }}\vspace{3in}}
\date{}
\author{\textbf{\authorName}}
%%%%%%%%%%%%%%%%%%%%%%%%%%%%%%%%%%%%%%%%%%%%%%%%%%%%%%%%%%%%%

\begin{document}
\begin{spacing}{1.1}

% Title page
%\maketitle
%\newpage

\section{The Problem}

The data set we are given is a video stream of a goldfish within a fish
tank. The fish was injected with a very high level of testosterone, with
the hypothesis that the increased testosterone levels would affect its
sexual behavior. The fish tank is directly bordering the tank of a
female, with a glass wall separating the two fish. The researchers
postulate that the goldfish will spend more time on the left end of the
tank, facing the female when it is sexually stimulated.

To explore this hypothesis, researchers are looking to quantify how much
time the goldfish spends on the left side of the tank, facing towards
the female fish. The image stream we are given is a black and white
video of the left side of the tank (the fish may leave the frame at any
point). The video is quite long, and of a high frame rate.

The researchers were not successful in using other computer vision
software tools. Our task is to develop a specialized tool to quantify
the goldfish's position and its direction.

\section{Information to Extract}

To accomplish this, the program must be able to first determine if the
goldfish is in fact in the frame. It then must be able to judge the
orientation of the goldfish in some manner. This makes the problem both
one of feature detection and classification, and of orientation. The
goldfish must then be able to be detected at any orientation, and have
some test to see if it is facing the female.

\section{Approach}

To determine if the goldfish is in the frame, we'll use a simple
background subtraction  To determine the orientation, we will use a
particle filter similar to the one described by Nummiero et. al in 2003
to track two locations on the fish. We will use an intensity histogram
as opposed to the color histogram to define an object. Then, determine
the probability of an object being in a location by using weighted
sampling and comparing the Bhattacharyya distance of the intensity
distribution at a given sample area. Then we can calculate the
horizontal distance between the most probable locations. If this
distance falls in an acceptable range, we can conclude that the fish is
facing to the left.

\subsection{Background subtraction}

The background subtraction method is quite simple. We merely take a
frame where the goldfish is not present, and subtract the given frame.
If the resulting image consists of zeroes, we can conclude that we have
another background frame, and thus, that the goldfish is not present.

\subsection{Particle filtering}

\end{spacing}
\end{document}
%%%%%%%%%%%%%%%%%%%%%%%%%%%%%%%%%%%%%%%%%%%%%%%%%%%%%%%%%%%%%
